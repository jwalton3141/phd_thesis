\thispagestyle{empty}

\clearpage

\vspace*{\fill}

\begin{center}
  \textbf{Abstract}\\[0.75cm]

  \begin{minipage}{0.9\textwidth}
    \setlength{\parskip}{0.45em}

    The study of collective behaviour---broadly defined as the formation of
    macro-level structures from the interactions between individuals---has in
    recent years become a thriving topic of multi-disciplinary research. Under
    consideration of biological-fitness the scientist has been able to reason
    about \emph{why} these structures form, and the advantages which the
    collective can afford the individual. However, much less is known about
    \emph{how} these structures are formed and maintained in the first place.

    Much work has been invested in the development of mathematical models which
    seek to explain the formation and maintenance of animal aggregations.
    Research has shown that behaviour reminiscent of real flocking events can
    arise from simple mathematical models which describe how individuals
    interact with one another. However, much of this modelling relies on
    aprioristic assumptions about how individuals behave and interact, with
    little-to-no verification against real observation.

    In this work we examine mathematical models popular in the literature and
    suggest modifications motivated by considerations of biological-realism. In
    particular we advocate adoption of continuous interaction rules, and
    consider how behavioural and biological variation can be accounted for by
    imposing hierarchical structure.

    We proceed to fit these models of collective behaviour to observations of
    real and simulated flocking events. Model fitting is performed in a
    Bayesian framework, allowing the quantification of parameter uncertainty.
    Fitting models to simulated data provides opportunity to assess the
    effectiveness and accuracy of our inference schemes, before attempting the
    same inference on real observation. Multiple competing models are fit to
    the same data, with the predictive performance of these models ranked using
    ideas from the model-selection literature. We are then able to recommend a
    subset of the candidate models as providing the best performance.

    Finally, consideration is made for datasets which exhibit missing
    observations. Such missingess occurs naturally owing to the fixed-location
    recording equipment used to capture flocking events. We argue that this
    missingness \emph{cannot} be ignored, and that it must be accounted for
    during any model-fitting process. Techniques are outlined which allow the
    researcher to account for missingness. Simulation studies are performed
    which demonstrate the efficacy of the proposed approach, before these
    techniques are demonstrated on a real dataset exhibiting missingness.

  \end{minipage}

\end{center}

\vspace*{\fill}

\cleardoublepage
