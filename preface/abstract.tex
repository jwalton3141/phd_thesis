\thispagestyle{empty}

\clearpage
\vspace*{\fill}

\begin{center}
  \textbf{Abstract}\\[0.75cm]

  \begin{minipage}{0.8\textwidth}
    The study of collective behaviour---broadly defined as the formation of
    macro-level structure from the interactions between individuals---has in
    recent years become a thriving topic of multi-disciplinary research. Under
    the guise of biological-fitness we are able to reason about \emph{why}
    these structures form, and the advantages which the collective affords the
    individual. However, much less is known about \emph{how} these structures
    are formed and maintained.

    Much work has been invested in developing mathematical models which attempt
    to explain and understand the formation and maintenance of these
    structures. Research has shown that behaviour reminiscent of real flocking
    events can arise from simple mathematical models which describe the
    interactions between individuals. However, much of this modelling relies
    on aprioristic assumptions about how individuals behave, with little-to-no 
    verification against real observation.

    In this work we examine models popular in the literature and suggest
    modifications motivated by considerations of biological-realism. In
    particular we advocate adoption of continuous interaction rules, and
    consider how behavioural and biological variation can be accounted for by
    imposing hierarchical structure.

    We proceed to fit models of collective behaviour to observations of real
    and simulated flocking events. Model fitting is performed within a Bayesian
    framework, allowing the quantification of parameter uncertainty. Fitting
    models to simulated data provides opportunity to assess the effectiveness
    and accuracy of our inference schemes, before attempting the same inference
    on real observation. Multiple competing models are fit to the same dataset,
    with the predictive performance of these models compared using ideas from
    the model-selection literature. We are then able to recommend a subset of
    the candidate models as providing the best performance.

    Finally, consideration is made for datasets which exhibit missing
    observations. Such missingess occurs naturally owing to the experimental
    set-ups used to record flocking events.
  \end{minipage}

\end{center}

\vfill
\cleardoublepage
