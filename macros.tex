%------------------------------------------------------------------------------
% Array strut to make delimiters come out right size both ends.
\newsavebox{\astrutbox}
\sbox{\astrutbox}{\rule[-5pt]{0pt}{20pt}}
\newcommand{\astrut}{\usebox{\astrutbox}}
%------------------------------------------------------------------------------

%------------------------------------------------------------------------------
% Redefine amsmath specification of cases to add a little padding after the 
% brace (https://tex.stackexchange.com/a/173086)
\makeatletter
\def\env@cases{%
  \let\@ifnextchar\new@ifnextchar
  \left\lbrace
  \def\arraystretch{1.2}%
  \array{l@{\quad}l@{}}% Formerly @{}l@{\quad}l@{}
}
\makeatother
%------------------------------------------------------------------------------

%------------------------------------------------------------------------------
% cdot looks too small for products, make something a little bigger
\makeatletter
\newcommand*\bigcdot{\mathpalette\bigcdot@{.5}}
\newcommand*\bigcdot@[2]{%
\mathbin{\vcenter{\hbox{\scalebox{#2}{$\m@th#1\bullet$}}}}}
\makeatother
%------------------------------------------------------------------------------

%------------------------------------------------------------------------------
% Stats stuff
\newcommand{\Ga}{\textrm{Ga}}
\newcommand{\Normal}{N}
% Use given in conditionals etc.
\newcommand{\given}{\,|\,}
% atan2 command
\DeclareMathOperator{\atantwo}{atan2}
%------------------------------------------------------------------------------

%------------------------------------------------------------------------------
% Bracket references to subfigures
\renewcommand\thesubfigure{(\alph{subfigure})}
%------------------------------------------------------------------------------

%------------------------------------------------------------------------------
\newcommand{\angmean}[1]{\langle#1\rangle}
%------------------------------------------------------------------------------

%------------------------------------------------------------------------------
% \bar command looks too small and \overline looks to big! Define \xoverline
% which is in the middle of the two. Use as \xoverline[width percent]{symb}
% Note that this will not scale inside sub or superscript
\makeatletter
\newsavebox\myboxA
\newsavebox\myboxB
\newlength\mylenA

\newcommand*\xoverline[2][0.75]{%
    \sbox{\myboxA}{$\m@th#2$}%
    \setbox\myboxB\null% Phantom box
    \ht\myboxB=\ht\myboxA%
    \dp\myboxB=\dp\myboxA%
    \wd\myboxB=#1\wd\myboxA% Scale phantom
    \sbox\myboxB{$\m@th\overline{\copy\myboxB}$}%  Overlined phantom
    \setlength\mylenA{\the\wd\myboxA}%   calc width diff
    \addtolength\mylenA{-\the\wd\myboxB}%
    \ifdim\wd\myboxB<\wd\myboxA%
       \rlap{\hskip 0.5\mylenA\usebox\myboxB}{\usebox\myboxA}%
    \else
        \hskip -0.5\mylenA\rlap{\usebox\myboxA}{\hskip 0.5\mylenA\usebox\myboxB}%
    \fi}
\makeatother
%------------------------------------------------------------------------------
