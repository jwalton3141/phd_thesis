\graphicspath{{fig/bayes_intro/}}

\chapter{Bayesian statistics}
\label{cha:bayes_intro}

In this thesis we utilise techniques from Bayesian inference to fit mathematical models of
collective behaviour to real data. Bayesian inference allows a practitioner to capture
uncertainty about fitted model parameters. In addition to this, the Bayesian framework
permits flexible model structures and potential inclusion of expert information via the
prior distribution. With this we seek to fit newly acquired data to generalisations of a
popular agent-based model from the literature.

In this chapter we shall introduce and give overviews of some important concepts of
Bayesian inference, outline schemes which can be used to infer model parameters, and
perhaps most importantly, discuss when our methodologies may fail us.

\section{Bayesian inference}
\label{sec:bayesian_inference}

Having observed data $\bm{x}$ we wish to quantify beliefs and uncertainties about
parameters $\bm{\theta} = (\theta_1, \theta_2,\dots,\theta_p)^T$. Given the observed data,
the likelihood function of the parameters is defined as:
\begin{equation}
  L(\bm{\theta}\given\bm{x}) = f(\bm{x} \given \bm{\theta}).
\end{equation}
The likelihood details the probability density of the data in terms of the parameters. We
may then specify our prior knowledge about the parameters $\bm{\theta}$ through the prior
distribution $\pi(\bm{\theta})$. Bayes Theorem can then be used to form our posterior
beliefs from the likelihood function and our prior beliefs:
\begin{equation}
  \label{eq:bayes_theorem}
  \pi(\bm{\theta}\given\bm{x}) =
    \frac{\pi(\bm{\theta}) L(\bm{\theta}\given\bm{x})}%
         {\int_{\bm{\theta}} \pi(\bm{\theta}) L(\bm{\theta}\given\bm{x}) \, \textup{d}\bm{\theta}}.
\end{equation}
As the integral in the denominator is not a function of $\bm{\theta}$ we may consider it a
constant of proportionality. With this we can express our posterior beliefs as
proportional to the product of the likelihood and our prior beliefs:
\begin{align*}
  \pi(\bm{\theta}\given\bm{x}) & \propto \pi(\bm{\theta}) \times L(\bm{\theta}\given\bm{x}), \\
  \text{posterior}             & \propto \text{prior} \times \text{likelihood}.
\end{align*}

\section{Markov chain Monte Carlo (MCMC)}
\label{sec:mcmc}

For the most part, the normalising constant (given in the denominator of
\cref{eq:bayes_theorem}) will have multiple dimensions, not produce a density function of
standard form, and be difficult to evaluate in all but the most trivial cases. Markov
chain Monte Carlo algorithms provide methods to sample from the targeted density
$\pi(\bm{\theta} \given \bm{x})$, whilst avoiding evaluating the bothersome normalising
constant.

\subsection{Metropolis-Hastings}
\label{ssec:metropolis_hastings}

The Metropolis-Hastings algorithm is a popular MCMC scheme. The algorithm was introduced
by \textcite{metropolis53} in a now classic paper, and was later generalised by
\textcite{hastings70}. The algorithm works by constructing a Markov chain which has
stationary distribution equivalent to the target distribution.

\begin{algorithm}
  \caption{Implement Metropolis-Hastings for $n$ iterations to target $\pi(\bm{\theta} \given \bm{x})$.}
  \label{alg:metropolis_hastings}
  \begin{algorithmic}[1]
    \State Initialise chain with $\bm{\theta}^{(0)}$
    \For{i}{1}{n}
      \State Propose ${\bm{\theta}}^\star \sim q(\bm{\theta}^{(i)} \given \bm{\theta}^{(i-1)})$
      \State Construct acceptance probability $\alpha(\bm{\theta}^\star \given \bm{\theta}^{(i-1)})$ as
      \begin{equation*}
          \alpha(\bm{\theta}^\star \given \bm{\theta}^{(i-1)}) =
        \min\bigg\{1,
        \frac{\pi(\bm{\theta}^\star) \, L(\bm{\theta}^\star \given \bm{x})}%
        {\pi(\bm{\theta}^{(i-1)}) \, L(\bm{\theta}^{(i-1)} \given \bm{x})}
        \frac{q(\bm{\theta}^{(i-1)} \given \bm{\theta}^\star)}%
        {q(\bm{\theta}^\star \given \bm{\theta}^{(i-1)})}
        \bigg\}.
      \end{equation*}
      \State Draw $u \sim$ Uniform$(0, 1)$
      \If{$u < \alpha(\bm{\theta}^\star \given \bm{\theta}^{(i-1)})$}
        \State \Comment{Accept proposal}
        \State $\bm{\theta}^{(i)} \leftarrow \bm{\theta}^\star$
      \Else
        \State \Comment{Reject proposal}
        \State $\bm{\theta}^{(i)} \leftarrow \bm{\theta}^{(i-1)}$
      \EndIf
    \EndFor
  \end{algorithmic}
\end{algorithm}

The algorithm begins by initialising a Markov chain with parameters $\bm{\theta}^{(0)}$.
Next, the algorithm proposes new parameter values $\bm{\theta}^\star$ from a proposal
distribution $q(\bm{\theta}^\star \given \bm{\theta}^{(i-1)})$. These proposed values are
accepted with probability $\alpha(\bm{\theta}^\star \given \bm{\theta}^{(i-1)})$.  If the
proposal is accepted the next state of the Markov chain is set to the proposed values,
otherwise the next state is set to the current values.  The acceptance probability depends
on a ratio of the posterior density evaluated at the current values and the posterior
density evaluated at the proposed values.  Because of this, the normalising constants
cancel in this ratio and we see that the target distribution only need be
known up to a constant of proportionality.  This process of proposing and accepting or
rejecting proposals continues until a satisfactory number of draws are made.
Metropolis-Hastings is described more formally in \cref{alg:metropolis_hastings}.

\subsubsection{Choosing a Proposal Distribution}
\label{ssec:proposal_distribution}

The practitioner must choose a suitable proposal distribution $q(\bm{\theta}^\star \given
\bm{\theta})$. Ideally the choice of proposal distribution will give rapid convergence to
$\pi(\bm{\theta} \given \bm{x})$ and efficiently explore the support of $\pi(\bm{\theta}
\given \bm{x})$. A special case of Metropolis-Hastings arises when the proposal
distribution is symmetric, that is
\begin{equation*}
  q(\bm{\theta}^\star \given \bm{\theta}) = q(\bm{\theta} \given \bm{\theta}^\star).
\end{equation*}
In this case we observe cancellation in the acceptance ratio, as it simplifies to become
\begin{equation*}
  \alpha({\bm{\theta}}^\star \given \bm{\theta}^{(i-1)})
    = \text{min}\,\bigg\{1,
                        \frac{\pi(\bm{\theta}^\star)}{\pi(\bm{\theta}^{(i-1)})}%
                        \frac{L(\bm{\theta}^\star \given \bm{x})}%
                        {L(\bm{\theta}^{(i-1)} \given \bm{x})}
                  \bigg\}.
\end{equation*}

The random walk sampler is a popular implementation of Metropolis-Hastings which makes
use of symmetric proposals. With this sampler proposals are realised as
\begin{equation*}
  \bm{\theta}^\star = \bm{\theta}^{(i-1)} + \bm{\omega}^{(i-1)},
\end{equation*}
where the $\bm{\omega}$ are sampled as
\begin{equation*}
  \bm{\omega}^{(i-1)} \sim \mathcal{N}_p(\bm{0}, \Sigma),
\end{equation*}
and $\mathcal{N}_p$ denotes a $p$-dimensional multivariate normal distribution. The
parameter $\Sigma$ is called the tuning parameter and controls how the chain moves around
the parameter space. Mixing describes how efficiently the chain moves around the sample
space and how long it takes for the chain to converge to the target distribution.

Crucially then, the parameter $\Sigma$ can be used to control the mixing of chains. So,
naturally, we wish to identify some `optimum' $\Sigma$ to try and improve mixing. Such a
tuning parameter should allow rapid convergence to $\pi(\bm{\theta} \given \bm{x})$ and
facilitate exploration of the entire support of the target. If the target distribution is
Gaussian, it has been shown that 0.234 is an optimum acceptance probability to try achieve
\parencite{roberts01}. In an attempt to tune $\Sigma$ to obtain the optimum acceptance
probability, a common technique is to use
\begin{equation*}
  \Sigma = \frac{2.38^2}{p} \widehat{\text{Var}}(\bm{\theta} \given \bm{x}).
\end{equation*}

However, even with strategies to try select some optimum innovation structure, random walk
samplers tend to perform poorly in high-dimensional problems. Consider that as the
dimension of a problem increases, the probability of proposing a point out in the tails of
the target distribution increases. As a result the acceptance probability becomes small and
produces a Markov chain which rarely moves. The acceptance probability can be increased
by choosing a $\Sigma$ which results in smaller innovations. However, this has the
consequence of producing a Markov chain which explores the same space slowly, and
converges to the target distribution slowly.

Fortunately, there exist more sophisticated proposal mechanisms which perform better than
random walk samplers in higher dimensional problems. One such sampler is represented by
Hamiltonian Monte Carlo, which seeks to utilise information about the gradient of the
target distribution to inform innovations.

\subsection{Hamiltonian Monte Carlo (HMC)}

Hamiltonian Monte Carlo, originally Hybrid Monte Carlo, was first introduced by
\textcite{duane87}. In this now landmark paper, the HMC algorithm was detailed and used
for numerical simulation of Lattice Quantum Chromodynamics. Following this, Radford Neal
recognised the potential statistical applications of HMC, and used it in his work on
Bayesian neural network models \parencite{neal95}. However, it wasn't really until Neal's
2011 review \parencite{neal11} that HMC received mainstream attention in statistical
computing \parencite{betancourt18}.

Hamiltonian Monte Carlo is a realisation of the Metropolis-Hastings algorithm. Here,
new parameter values are proposed by computing trajectories of motion according to
Hamiltonian dynamics. With this proposal mechanism it is possible to propose parameter
values which are distant from the current state, but which retain a high probability of
acceptance. As a result, this proposal mechanism represents an efficient method of
traversing the parameter space, and circumvents the slow exploration of the parameter
space typically experienced by random walk samplers in higher-dimensions.

\subsubsection{Mathematical formulation}

Hamiltonian mechanics represent a reformulation of classical mechanics, where a system is
described by a $d$-dimensional position vector, $q$, and a $d$-dimensional momentum
vector, $p$. This system then evolves through time according to Hamilton's
equations:
\begin{align}
  \begin{split}
    \pdv{p_i}{t} &= - \pdv{\mathcal{H}}{q_i}\\
    \pdv{q_i}{t} &= \hphantom{-} \pdv{\mathcal{H}}{p_i}
  \end{split}
\end{align}
where $i=1,\ldots,d$ and $\mathcal{H}(p, q)$ is the Hamiltonian. The
Hamiltonian is often interpreted to represent the total energy of a system, which can be
considered as the sum of the kinetic energy, $T$, and potential energy, $V$:
\begin{equation}
  \label{eq:hamiltonian_decomp}
  \mathcal{H}(q, p) = T(p) + V(q).
\end{equation}

We wish to explore our target distribution (typically the posterior distribution) as if
evolving some Hamiltonian system. This can be achieved if we expand our $d$-dimensional
parameter space into $2d$-dimensional phase space. Our current state can be considered as
the position vector, $q$. Introducing auxiliary momentum variables, $p$, expands
our parameter space into phase space, as desired.

With our parameter space extended to phase space, we must also expand our target
distribution to phase space. To do so we formulate the canonical distribution, a joint
density function over phase space:
\begin{equation}
  \label{eq:canonical_dist}
  \pi(q, p) = \pi(p \given q) \, \pi(q).
\end{equation}
The momentum is typically introduced as:
\begin{equation}
  \pi(p \given q) = \exp{-\frac{p^T M^{-1} p}{2}},
\end{equation}
where $M$ is a positive-definite ``mass matrix'', often chosen as the identity matrix or
some scalar multiple of the identity matrix. See that marginalising out the momentum in
\cref{eq:canonical_dist} recovers the target distribution.

To proceed, we consider expressing the canonical distribution as the
negative exponent of a Hamiltonian:
\begin{equation}
  \label{eq:canonical_as_hamiltonian}
  \pi(q, p) = \exp{-\mathcal{H}(q,p)}.
\end{equation}
Taking the logarithm of \cref{eq:canonical_as_hamiltonian} and using \cref{eq:canonical_dist}
we see
\begin{equation}
  \label{eq:canonical_decomp}
  \mathcal{H}(q, p) = -\log{\pi(p \given q)} - \log{\pi(q)}.
\end{equation}
Recall from \cref{eq:hamiltonian_decomp} that the total energy in a system can be
considered as the sum of the system's kinetic energy and potential energy.  If we compare
\cref{eq:hamiltonian_decomp} and \cref{eq:canonical_decomp} we can see that we have
constructed a system with kinetic energy given by the negative logarithm of the momentum
density, and potential energy given by the negative logarithm of the target density.

\subsubsection{Computer implementation, NUTS \& Stan}

Now that we have described HMC we are in a position to consider its implementation
\emph{in silico}. For this computer implementation we must first be able to approximate
solutions to Hamilton's equations. Such approximations can be achieved by discretising
time using some small time step $\epsilon$. Next, the practitioner must also choose the
number of steps $L$ for which to simulate Hamilton's equations. With this in place, the
practitioner typically implements the leapfrog method to solve Hamilton's equations
\parencite{neal11}.

As we have seen, in implementing HMC it is left to the practitioner to choose appropriate
values for $L$ and $\epsilon$. Unfortunately, making a poor choice for either of these
parameters can result in a significant decrease in the performance of HMC
\parencite{hoffman14}. Actually, $\epsilon$ can be tuned during the algorithm's
implementation, using ideas from the adaptive MCMC literature. However, there is no easy
way to select a value of $L$ \emph{a priori}. Typically, a practitioner will have to make
multiple costly tuning runs in order to select an appropriate value of $L$.

It was with this tuning problem in mind that \textcite{hoffman14} introduced the No-U-Turn
Sampler (NUTS). This algorithm extends HMC and eliminates the need for the parameter $L$.
Using primal-dual averaging the authors were also able to tune $\epsilon$ during the
implementation of the algorithm. Altogether then, NUTS represents an implementation of HMC
where the practitioner is left free of the obligation of choosing any tuning parameters.

Although NUTS relieves the practitioner the obligation of selecting parameters $\epsilon$
and $L$, its implementation remains far from trivial. Here enters Stan. Stan, named after
Stanislaw Ulam, one of the original pioneers of Monte Carlo methods
\parencite{metropolis49}, is a probabilistic programming language implemented in C++
\parencite{gelman15}. Stan requires the user to construct a Stan program, which specifies
how to compute the log-posterior density for their model of interest
\parencite{stanteam15}. With this, Stan implements the NUTS algorithm, and returns to the
user samples from the desired posterior density.

\subsection{Convergence Diagnostics}
\label{ssec:convergence_diagnostics}

Though there are theoretical methods to assess the convergence of chains, it is an attractive idea to
analyse the output of our schemes in an attempt to assess whether the chains have converged. One of
the simplest informal methods to assess convergence is to inspect the trace plots of the scheme and
check for any irregularities. It is also good to use autocorrelation plots to assess autocorrelation
between samples at different lags.

One way to lower autocorrelation between samples is to thin the output. When thinning, every $k$-th
sample from a chain is kept and the remaining samples are discarded. Another common technique is to
allow for a burn-in period. The purpose of a burn-in period is to discard any samples from before the
chain has converged.

\section{Model selection}
\label{sec:model_comparison}
