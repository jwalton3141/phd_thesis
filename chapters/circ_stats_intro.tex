\graphicspath{{fig/circ_stats_intro/}}

\chapter{Directional statistics}	
\label{cha:direct_stats}

Circular data arises naturally in the study of collective behaviour; most
commonly, in describing the direction of motion of individuals. Given some
dataset, the first instinct of the scientist is to summarise and visualise the
data. However, such a researcher should proceed with caution: circular data
cannot be treated as if it were its linear counterpart.

In this appendix we shall consider why standard techniques, methods and
summaries are inappropriate to use with circular data. After this realisation,
we proceed to introduce some useful techniques which can be used to handle and
visualise directional data.

\section{Conventions}
\label{sec:conventions}

Directions can be represented as rotations with respect to some zero-direction,
or origin. The practitioner is free to chose the zero-direction as they feel
appropriate. In a similar way, the practitioner may choose whether a clockwise
or anti-clockwise rotation is taken as the positive direction.

Recall that angles may be represented in units of degrees or radians. To
convert between degrees and radians we may multiply by a factor of
$\pi/\ang{180}$.

In this thesis we define the zero-direction as the direction from the point
$(0, 0)$ along the positive $x$-axis. For the most part, we shall measure
angles in units of radians, and take anti-clockwise rotations as the positive
direction. The schematics of this setup are illustrated in
\cref{subfig:radian_axes}. Occasionally, we shall appeal to degrees and their
comparative intuitiveness, and in these cases we shall use the setup
illustrated in
\cref{subfig:degree_axes}.

\begin{figure}[tb]
\centering
    \begin{subfigure}[b]{0.5\textwidth}
		\includegraphics{radian_axes.pdf}
		\caption{}
		\label{subfig:radian_axes}
	\end{subfigure}%
	\begin{subfigure}[b]{0.5\textwidth}
		\centering
		\includegraphics{degree_axes.pdf}
		\caption{}
		\label{subfig:degree_axes}
	\end{subfigure}
    \caption{Visualising the conventions used in this thesis to measure
      \subref{subfig:radian_axes}: radians and \subref{subfig:degree_axes}:
      degrees.}
	\label{fig:compare_axes}
\end{figure}

\section{Visualisation}
\label{sec:circular_visualisation}

In possession of a dataset, one of the first instincts of the scientist is to
visualise their data. The researcher is undoubtedly familiar with a large
number of graph types. Yet choosing the most suitable graph to display a given
dataset is crucial in making an informative plot.

Traditional histograms are not very good for visualising directional data;
consider, for example, interpreting the directions plotted in
\cref{fig:angle_hist}. Polar histograms (sometimes known as rose plots) make
for more intuitive representations of angles. Instead of using bars, as the
histogram does, the rose plot bins data into sectors of a circle. Here, the
\emph{area} of each sector is constructed to be proportional to the frequency
of data points in the corresponding bin \parencite{mardia09}.

\begin{figure}[tb]
	\begin{subfigure}[b]{0.5\textwidth}
		\includegraphics{uniform_hist.pdf}
		\caption{}
		\label{subfig:unif_angle_hist}
	\end{subfigure}%
	\begin{subfigure}[b]{0.5\textwidth}
		\includegraphics{normal_hist.pdf}
		\caption{}
		\label{subfig:norm_angle_hist}
	\end{subfigure}
    \caption{Using histograms to visualise \subref{subfig:unif_angle_hist}: one
      hundred samples drawn from $U(-\pi, \pi)$ and
      \subref{subfig:norm_angle_hist}: ten thousand samples drawn from $N(0, 1)$.}
	\label{fig:angle_hist}
\end{figure}

\begin{figure}[tb]
	\begin{subfigure}[b]{0.45\textwidth}
		\includegraphics{uniform_polar_hist.pdf}
		\caption{}
		\label{subfig:unif_angle_rose}
	\end{subfigure}%
	\hspace{0.05\textwidth}%
	\begin{subfigure}[b]{0.45\textwidth}
		\includegraphics{normal_polar_hist.pdf}
		\caption{}
		\label{subfig:norm_angle_rose}
	\end{subfigure}
    \caption{Using polar histograms to visualise
      \subref{subfig:unif_angle_hist}: one hundred samples drawn from $U(-\pi,
      \pi)$ and \subref{subfig:norm_angle_hist}: ten thousand samples drawn from
      $N(0, 1)$.}
    \label{fig:angle_rose}
\end{figure}
To advocate the advantages of the rose plot we shall visualise two randomly
generated datasets. The first dataset consists of one hundred realisations from
a uniform $U(-\pi,\pi)$ distribution, and the second dataset consists of ten
thousand draws from a normal $N(0, 1)$ distribution.

In \cref{fig:angle_hist} we visualise the two datasets using traditional
histogram plots. From this figure we get a good idea of the distribution of the
data, however we get no sense of direction.  In \cref{fig:angle_rose} we
visualise the same data. Here we also get a good idea of how the directions are
distributed. However, using the rose plot means we get a very intuitive
representation of direction. 

\section{Summary statistics}
\label{sec:summary_stats}

Summary statistics are a useful tool to give an idea of the general
characteristics of a dataset. Probably the first statistic which we learn to
compute is the arithmetic mean. The arithmetic mean, however, is not an
appropriate statistic to use with circular data.

Consider that we wish to take an average of the angles $\ang{10}$ and
$\ang{350}$. Using the arithmetic mean we compute an average of $\ang{180}$.
However, this average points in the opposite direction to which we intuitively
expect. In \cref{subfig:arith_mean} we visualise this result.

\begin{figure}
	\begin{subfigure}[b]{0.5\textwidth}
		\includegraphics{arith_mean.pdf}
		\caption{Arithmetic mean (red).}
		\label{subfig:arith_mean}
	\end{subfigure}%
	\begin{subfigure}[b]{0.5\textwidth}
		\includegraphics{circ_mean.pdf}
		\caption{Circular mean (green).}
		\label{subfig:circ_mean}
	\end{subfigure}
    \caption{Computing the average of $\ang{10}$ and $\ang{350}$ (represented
      by the blue arrows), using two different mean functions. The green and red
      arrows show the average computed by each method.}
	\label{fig:visualise_mean}
\end{figure}

Before introducing the circular mean it is first necessary to introduce the
$\atantwo$ function. The $\atantwo$ function dates back to the Fortran
programming language \parencite{organick66}. It was introduced to overcome some
of the inconveniences inherent in the $\atan$ (or $\tan^{-1}$) function.
Consider that the inverse tangent function has codomain $(-\pi/2, \pi/2)$,
though we are often interested in directions in the range $(-\pi, \pi]$. In
addition to this, the $\arctan$ function is not quadrant-aware; that is, it
cannot distinguish between directions which differ by $\pi$ radians (see that
$\tan^{-1}(\theta + \pi) = \tan^{-1}(\theta))$.  As an example, consider
calculating the direction from the $x$-axis to the ray extending from the
origin to the point $(1, 1)$. Naturally, we'd reach for $\tan^{-1}$ to compute
the angle as $\tan^{-1}(1/1) = \pi/4$, as expected. Now, consider that we wish
to calculate the direction from the $x$-axis to the ray extending from the
origin to the point $(-1, -1)$. By inspection, or intuitively, we expect an
answer of $-3\pi/4$ --- however, we compute the answer as $\tan^{-1}(-1/-1) =
\pi/4$. The angle calculated using the inverse tangent function points in the
opposite direction to what we expect.

\begin{figure}[tb]
	\includegraphics{atan_quadrants.pdf}
	\caption{An illustration of the quadrant corrections made by $\atantwo$.}
	\label{fig:atan_quadrants}
\end{figure}

The $\atantwo$ function, however, does \emph{not} have these shortcomings. The
function is constructed to be quadrant-aware: correcting the computations of
$\tan^{-1}$ to return the directions we intuitively expect. It does so by
adding a correction term which depends on the quadrant which contains our point
of interest $(x, y)$. The correction term applied in each of the four quadrants
is visualised in \cref{fig:atan_quadrants}. With these considerations,
$\atantwo$ can be realised by the piecewise function:

\begin{equation}
\label{eq:atantwo}
	\atantwo(y, x) = 
	\begin{cases}
		\atan(y/x)          & \text{ if } x > 0, \\
		\atan(y/x) + \pi    & \text{ if } x < 0 \text{ and } y \geq 0, \\
		\atan(y/x) - \pi    & \text{ if } x < 0 \text{ and } y < 0, \\
		\pi/2               & \text{ if } x = 0 \text{ and } y > 0, \\
		-\pi/2              & \text{ if } x = 0 \text{ and } y < 0, \\
		\text{undefined}    & \text{ if } x = 0 \text{ and } y = 0. \\
	\end{cases}
\end{equation}

As we saw in \cref{subfig:arith_mean}, averaging a set of angles with the
arithmetic mean does not give the desired result. Instead, we must refer to the
circular mean. Given a set of angles $\theta = (\theta_1, \ldots, \theta_n)^T$,
we may compute their circular mean as:
\begin{equation}
	\label{eq:circ_mean}
	\langle \theta \rangle = \atantwo\bigg(\frac{1}{n} \sum_{j=1}^n \sin(\theta_j), \frac{1}{n} 
    \sum_{j=1}^n \cos(\theta_j)\bigg),
\end{equation}
where the $\atantwo$ function is defined in \cref{eq:atantwo}
\parencite{fisher95}.

The definition of the circular mean given in equation \cref{eq:circ_mean} works
by converting the angles into Cartesian co-ordinates: representing the
directions as points on the unit circle. The centre of mass of the Cartesian
co-ordinates is then computed, and the resulting position is converted back to
a direction, resulting in our mean angle.

In practice, the $1 / n$ which occurs in \cref{eq:circ_mean} is superfluous.
Referring to \cref{eq:atantwo}, see that all of the cases involving $\atan$
require the quotient of $x$ and $y$. Because of this ratio, the $1 / n$ terms
will always cancel, and so aren't strictly necessary.

\section{von Mises distribution}

The von Mises distribution, sometimes simply referred to as the circular normal
distribution, is a continuous probability density function defined on the
circle, with support $[-\pi, \pi)$. The distribution is parameterised by two
parameters: $\mu \in [-\pi, \pi)$ and $\kappa > 0$. The parameter $\mu$ is a
measure of location and the parameter $\kappa$ is a measure of spread. These
parameters, $\mu$ and $\kappa$, are analogous to $\mu$ and $1/\sigma^2$ of the
normal distribution.

For the angle $\theta$, the von Mises distribution has probability density:
\begin{equation}
    \label{eq:von_mises}
	f(\theta \given \mu, \kappa) = \frac{e^{\kappa \cos(\theta - \mu)}}%
	                                    {2 \pi I_0(\kappa)},
\end{equation}
where the normalising constant, $I_0(\kappa)$, is the modified Bessel function
of the first kind and order zero \parencite{jammalamadaka01}.

In this thesis we do \emph{not} use the von Mises distribution. Instead, we
continue to use a normal distribution to model circular data. This
approximation is appropriate when $\kappa$ is large (that is, when there is
little dispersion). For large $\kappa$ it is known that $I_0(\kappa) \approx
e^k / \sqrt{2\pi\kappa}$. Using this, and the Taylor expansion $\cos(\alpha)
\approx 1 - \alpha^2/2$, from \cref{eq:von_mises} we have:
\begin{align*}
    f(\theta \given \mu, \kappa) &\approx \frac{e^{\kappa[1-\frac{1}{2}(\theta-\mu)^2]}}%
                                               {2\pi e^k / \sqrt{2\pi\kappa}}\\
                                 & = \frac{e^{-\frac{\kappa}{2}(\theta-\mu)^2}}%
                                          {\sqrt{2\pi/\kappa}},
\end{align*}
which is just the probability density of the normal distribution with mean
$\mu$ and precision $\kappa$. So we see, for distributions with small
dispersion, it is appropriate to approximate the von Mises distribution with a
normal distribution.

