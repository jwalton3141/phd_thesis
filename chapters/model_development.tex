\graphicspath{{fig/model_development/}}

\chapter{Model Development}
\label{cha:model_dev}

In this chapter we shall introduce one of the most popular and well studied agent-based
models in the literature: the Vicsek model. The Vicsek model represents a simple alignment
model in which agents interact with neighbours within some given distance. Despite its
simplicity, this model can produce sophisticated dynamics and exhibits a phase
transition from order to disorder as the amount of noise in the system is regulated.

The Vicsek model, like many other ABMs, implements a discontinuous interaction rule. With
this, the onset of interaction between individuals is very sensitive to small
perturbations in distances. We consider continuous interaction rules as a
biologically-realistic alternative. Such rules ensure the model is more robust to small
perturbations in distances, but without adding model complexity.

\section{The Vicsek Model}

The Vicsek model describes the movements of $N$ individuals moving with constant
speed $v$. All movement takes place in a square cell with side length $L$, and periodic
boundary conditions. To initialise a simulation all agents are allocated a random
position within the cell, and a random direction of motion. From time $t$ to time $t+1$
agent $i$ updates its position as:
\begin{equation*}
    \bm{x}_{i, t+1} = \bm{x}_{i, t} + \bm{v}_{i, t},
\end{equation*}
where the velocity $\bm{v}_{i,t}$ is constructed to have speed $v$ and direction of
motion:
\begin{equation*}
    \theta_{i, t+1} \given \angmean{\theta}_{i, t}, \eta \sim
                     \mathcal{U}(\angmean{\theta}_{i, t} - \eta/2,
                                 \angmean{\theta}_{i, t} + \eta/2),
\end{equation*}
where $\mathcal{U}$ is the uniform distribution, $\angmean{\theta}_{i, t}$
