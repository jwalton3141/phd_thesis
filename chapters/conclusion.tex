\chapter*{Conclusions}
\label{cha:conc}
\addcontentsline{toc}{chapter}{\nameref{cha:conc}}
\markboth{Conclusions}{Conclusions}

Mathematical modelling has become a cornerstone of the study of collective
behaviour, as the formation of macro-level structure was realised possible by
implementations of simple behavioural rules. Although model simulation can be
informative in its own right, its real power comes in comparison with
observation. Comparison of model prediction and real-world observation is an
essential part of any model-fitting and model-verification process. However,
until recently thorough comparison between collective behaviour model and data
has been limited by the availability of data of real flocking events.

In this work we acknowledged the technical challenges presented by the
recording of flocking events, and considered the affect of this on the
resulting literature. A lack of data with which to perform model-verification
has resulted in a plethora of competing models based on aprioristic assumptions
about how individuals behave and respond to stimulus. However, more recently
there have been notable attempts to bridge the gap between model and data. An
overview of this recent work, and work prior to it, was presented in
\cref{cha:lit_review}.

Backed by previous studies, we argued the importance of the quantification of
uncertainty during model-fitting. Bayesian inference, representing a
fully-probabilist approach to parameter inference, is a natural approach to
quantifying uncertainty. The underlying philosophy of the Bayesian paradigm was
outlined in \cref{cha:bayes_intro}. Here we also introduced Markov chain Monte
Carlo methods: a class of algorithms which allow a practitioner to draw samples
from the posterior distribution, whilst only requiring knowledge of the
posterior up to a constant of proportionality.

In fitting models to data it is advisable to first consider simple models, and
to only increase model-complexity when necessary. The Vicsek model in
\cref{cha:model_dev} was introduced as one such simple model, accounting for
directional changes as some combination of alignment interactions and noise. We
questioned the biological-realism the zonal interaction rule implemented by
Vicsek represents, instead advocating the merits of continuously differentiable
interaction rules. As well as biological-realism, we acknowledged the
difficulties which discontinuous interaction rules pose for parameter
inference.

Despite understandings of the importance of biological variation in nature, the
majority of models in the literature assume that all agents behave identically
when presented with the same stimulus. With this shortcoming in mind we
introduced models with hierarchical structure: allowing intra-flock behavioural
and biological variation.

\cref{cha:sim_studies} was a chapter of simulation studies. Simulation studies
are a useful tool in assessing the viability of performing parameter inference
on real data. The models introduced and developed in the preceding chapter
were forward simulated for known parameter values. We then demonstrated that
the Metropolis--Hastings and NUTS algorithms could be used to accurately infer
the true parameter values from observations of the simulated flocks alone.

Having demonstrated the efficacy of our approach to inference on simulated
data, we proceeded to re-use our statistical machinery to perform inference on
real data. In \cref{cha:sheep} we introduced a dataset consisting of three
recording events of flocking sheep. To this data we fit variations of the
Vicsek model as introduced in \cref{cha:model_dev}. The within-sample
predictive performance of these models were quantified with information
criteria. Models were then ranked by their performance. We observed that
continuous interaction rules consistently outranked their discrete counterpart,
as well observing evidence of non-normally distributed noise. In general the
models provided a good fit to data, however in one of the sequences there was
evidence of the existence of behaviours \emph{not} accounted for in our models.
We speculate that these behaviours could be the result of some unaccounted for
interactions with the external environment, or additional inter-individual
interactions such as attraction or repulsion.

As it is impossible to predict when and where a given flocking event may occur
next, there can become a frustrating ``right-place-right-time'' component to
data collection. Flocking events may be captured by establishing fixed-location
recording equipment where a researcher believes a flocking event may occur in
the future. However, the stationary nature of the recording equipment can
result in capturing incomplete flocking events. To evaluate the likelihood of
the data---a critical component of any model-fitting, Bayesian or
otherwise---the observation of every individual in the flock, at every recorded
frame, is required. A naive solution to this problem is to discard every frame
in which the entire flock isn't within frame. However, this approach has the
potential to drastically reduce the amount of data available for analysis. In
\cref{cha:missing} we presented an alternative approach which incorporates the
additional uncertainty introduced by missingness into our posterior
distributions. Simulation studies were performed which showed that our approach
consistently outperformed the naive approach.






