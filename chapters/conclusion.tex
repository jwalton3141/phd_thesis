\chapter*{Conclusions}

Mathematical modelling has become a cornerstone of the study of collective
behaviour, as the formation of macro-level structures can be realised by
implementations of simple behavioural rules. Although model simulation can be
informative in its own right, its real power comes in comparison with
observations of flocking events. Comparison of model prediction and real-world
observation is an essential part of any model-fitting and model-verification
process. However, in the past the ability to do so has been limited by the
difficulty in capturing and extracting data of real flocking observations.

In this thesis we demonstrated the applicability of Bayesian inference to
problems in collective behaviour. 

