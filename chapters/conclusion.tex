\chapter*{Conclusions}
\label{cha:conc}
\addcontentsline{toc}{chapter}{\nameref{cha:conc}}
\markboth{Conclusions}{Conclusions}

Mathematical modelling has become a cornerstone of the study of collective
behaviour, as the formation of macro-level structure was realised possible by
implementations of simple behavioural rules. Although model simulation can be
informative in its own right, its real power comes in comparison with
observations of real events. Comparison of model prediction and real-world
observation is an essential part of any model-fitting and model-verification
process. However, until recently thorough comparison between collective
behaviour model and data has been limited by the availability of data of real
flocking events.

In this work we acknowledged the technical challenges presented by the
recording of flocking events, and considered the affect of this on the
resulting literature. A lack of data with which to perform model-verification
has resulted in a plethora of competing models based on aprioristic assumptions
about how individuals behave and respond to stimulus. However, more recently
there have been notable attempts to bridge this gap between model and data. An
overview of this recent work, and work prior to it, was presented in
\cref{cha:lit_review}.

Backed by the work of others, we argued the importance of the quantification of
uncertainty during model-fitting. Bayesian inference, representing a
fully-probabilist approach to parameter inference, is a natural approach to
quantifying uncertainty. The underlying philosophy of the Bayesian paradigm was
outlined in \cref{cha:bayes_intro}, as were modern Bayesian methods for
performing parameter inference.

In fitting models to data it is advisable to first consider simple models, and
to only increase model-complexity when necessary. The Vicsek model in
\cref{cha:model_dev} was introduced as one such simple model, accounting for
directional changes as some combination of alignment interactions and noise. We
questioned the biological-realism the zonal interaction rule implemented by
Vicsek represented, instead advocating the merits of continuously
differentiable interaction rules. Despite understandings of the importance of
biological variation in nature, the majority of models in the literature assume
that all agents behave identically when presented with the same stimulus. With
this shortcoming in mind we introduced models with hierarchical structure:
allowing intra-flock behavioural and biological variation.

\cref{cha:sim_studies} was a chapter of simulation studies. Simulation studies
are a useful tool in assessing the viability of performing parameter inference
on real data. The models introduced and developed in the preceding chapter
were forward simulated for known parameter values. We then demonstrated that
the Metropolis--Hastings and NUTS algorithms could be used to accurately infer
the true parameter values from observations of the simulated flocks alone.

Having demonstrated the efficacy of our approach to inference on simulated
data, we proceeded to re-use this statistical machinery to perform inference on
real data. In \cref{cha:sheep} we introduced a dataset consisting of three
recording events of flocking sheep. To this data we fit variations of the
Vicsek model as introduced in \cref{cha:model_dev}. The within-sample
predictive performance of these models were quantified with information
criteria. Models were then ranked by their performance. We observed that
continuous interaction rules consistently outranked their discrete counterpart.
In general the models provided a good fit to data, however in one of the
sequences there was evidence of the existence of behaviours \emph{not}
accounted for in our models. We speculate that these movements could be the
result of some unaccounted for interactions with the external environment, or
additional inter-individual interactions such as attraction or repulsion.









