\graphicspath{{fig/scoters/}}

\chapter{Case Study II: foraging scoters}
\label{cha:scoters}

In the previous chapter we discussed how the use of fixed-location recording
equipment during data acquisition allows individuals to move out-of-frame
\emph{during} recording. Such movements are then missing from the recorded
flocking event. As the out-of-frame individuals may still be influencing the
behaviour of the observed flock, their movements must be accounted for during
any resulting analysis. We demonstrated methodology to account for this
missingness on \emph{simulated} data. In this chapter we consider a dataset
which details a \emph{real} flocking event with missing observations. As
foreshadowed in the previous chapter, the missing observations of this dataset
will be realised as a consequence of employing fixed-location recording
equipment.

The dataset considered in this chapter describes the movements of a large
number of surf scoters, a migratory sea bird, which gather in groups to forage.
The captured flocking events take place on the surface of a lake, and so
movement is effectively restricted to a two-dimensional plane. This dataset was
provided courtesy of work performed by \textcite{lukeman09,lukeman10}. At the
time the captured events represented a tenfold increase in the number of
individuals which could be reliably tracked \emph{between} frames.

Unfortunately, owing to the fixed-location cameras used to record these
flocking events, there were often scoters out of view during recorded frames.
\cref{fig:lukeman_extraction} illustrates this with frames captured during the
recording process. As in \cref{cha:missing}, the missing data points can be
partitioned into observations missing at the \emph{beginning} of the recording
event, and observations missing at the \emph{end} of the recording event.

In this chapter we shall fit a simple variation of the Vicsek model to a
flocking event of this dataset. The missing data points will be integrated out
of the problem, using the methodology developed in \cref{cha:missing}.

\section{Scoter data}

\textcite{lukeman10} captured their flocking events from an aerial vantage
point: an elevated promenade at the side of a large lake. From this position
the authors were able to direct a camera toward an inlet where overwintering
scoters had been observed foraging. Knowing the height of this camera above the
water, and the angle of its approach with the horizontal, the authors were able
to transform the captured camera data back to ``real-world'' co-ordinates
through trigonometry.

However, to reliably transform back from the camera co-ordinates to real-world
co-ordinates, the height of the camera above the lake and its angle with the
horizontal had to be fixed and known. With the camera fixed in position, the
authors then waited for flocking events to occur. Recording began when
individuals entered the camera's field of vision, and ended when all the
individuals left the frame, or the flock became stationary.

Each recording event captured the movements of $\approx170$ scoters, with each
event lasting for $\approx30$ frames. However, for these sequences between
$16\%$ and $64\%$ ($\approx800$--$3000$ observations) of the data points were
missing.

\textcite{lukeman10} discounted the influence of individuals out-of-frame, and
instead focussed on reproducing radial and nearest neighbour distributions of
internal group members (edge individuals were also discarded). This approach,
although representing a significant step forward in the literature, came with
its drawbacks. Most significantly, focussing the fitting on reproducing
nearest-neighbour and radial neighbour distributions removed the \emph{dynamic}
component to the data which the authors worked so hard to achieve.
Additionally, the authors determined the size of the agents' interaction radii
by visual inspection of nearest-neighbour densities. That is, the interaction
radii used in their model were chosen \emph{before} the fitting process began.

In this chapter we take a more holistic approach to model fitting; making sure
to account for \emph{all} individuals---both observed and missing, as well as
internal and boundary individuals---and focussing the fitting on reproducing
the \emph{movements} of the flock, rather than some epiphenomena of their
movements. Significant computation is necessary to account for the large amount
of missingness represented by this data.

\section{Performing inference}

In \cref{cha:missing} we discussed how the computational load of our inference
scheme increases as the number of missing observations increases. With this in
mind we will be focussing on an event within the Lukeman dataset which displays
the \emph{least} amount of missingness. For the events captured by
\textcite{lukeman10} this is represented by a sequence of $199$ scoters moving
over $23$ frames. Of the $199\times23=4577$ data points represented by this
event, $680$ ($\approx16\%)$ of the datapoints occur out of the visual field of
the camera. This particular event is visualised in \cref{fig:scoter_traj}. From
this figure we see that the flock was strongly polarised as the individuals
moved through the frame; it is with this in mind that we consider an
alignment-based model a plausible candidate model.

\begin{figure}[tb]
  \includegraphics{data_00_traj.pdf}
  \caption{A visualisation of the trajectories of foraging scoters in the
      analysed sequence. In total the sequence represents the movements of
      $199$ scoters over $23$ frames ($4577$ data points). The scoters move
      along the blue trajectories, starting from the positions denoted by the
      green markers and finishing at the positions shown by the red markers.
      The black frame containing the trajectories represents an approximation
      of the fixed field of vision of the recording equipment.}
  \label{fig:scoter_traj}
\end{figure}

\cref{fig:scoter_missing} gives a visual representation of which data points
were observed, and which data points were missing. In this visualisation blue
markers represent observed data and red markers represent missing data. From
\cref{fig:scoter_missing} we see that in \emph{every frame} there is \emph{at
least one} missing agent. As such, if we were to discard all the frames with
missing scoters (as in the naive approach outlined in \cref{cha:missing}), we
would actually end up discarding the entire dataset.

The simulated flocks considered in \cref{cha:missing} had between $10$ and $20$
missing observations per sequence. With $680$ data points missing from the
scoter sequence with the \emph{least} amount of missingness, this problem
represents a considerable increase in complexity. With this amount of
missingness, long and expensive simulations were necessary to realise a
satisfactory number of samples from the posterior.

\begin{figure}[tb]
  \includegraphics{data_00_missing.pdf}
  \caption{A representation of the missing and observed data points of the
    foraging event shown in \cref{fig:scoter_traj}. The $x$-axis represents the
    frame of the sequence, and the $y$-axis corresponds to tracked individuals. A
    blue tile at location $(i,t)$ tells us that scoter $i$ was observed in frame
    $t$. A red tile at location $(i,t)$ indicates that scoter $i$ was missing at
    time $t$.}
  \label{fig:scoter_missing}
\end{figure}

We fit a variation of the Vicsek model to this dataset, where each agent
interacts with neighbours within distance $r$, and experiences noise generated
from a generalised Student's $t$ distribution with $\nu$ degrees of freedom and
scale $\sigma_Y$. A random walk Metropolis--Hastings algorithm was implemented
to infer the parameters of the model. In a similar manner, a random walk
Metropolis--Hastings algorithm, with proposal scheme as outlined in
\cref{ssec:beg_missing}, was used to infer the observations missing at the
beginning of the sequence. Finally, using a Gibbs step within the
Metropolis--Hastings step allowed observations missing at the end of the
sequence to be sampled. This sampler was implemented for $10^8$ iterations. To
reduce the memory overhead of this sampler, the recorded samples were thinned
by a factor of $1000$. This means that all but every $1000$-th iteration of the
sampler were discarded, leaving us with $10^8 / 10^3 = 10^5$ iterations.

In \cref{cha:missing} we used summary plots---such as those in
\cref{fig:beg_summary,fig:end_summary}---to help visualise high-dimensional
posterior distributions. However, with $680$ dimensions (plus a few extra for
the model parameters), this posterior has too many dimensions for even these
summary plots. In such a situation it can be informative to graphically assess
the trace of the log-likelihood. If we can observe that the log-likelihood has
converged, we have good evidence that all the chains have also converged. In
\cref{fig:log_ll} we see a trace and histogram plot of values realised from the
log-likelihood. From this plot we see evidence of convergence, as the
log-likelihood oscillates regularly around some fixed location.

\begin{figure}[tb]
  \includegraphics{log_likelihood.pdf}
  \caption{Trace and histogram plots of the log-likelihood, as computed at
    every iteration of the inference scheme. As it can be difficult to assess
    the convergence of all the parameters for high-dimensional problems, as
    we have here, it can instead be informative to assess the convergence of
    the log-likelihood. If we see that the value of the log-likelihood has
    converged, then we have evidence that the chains targeting the corresponding
    parameters also converged.}
    \label{fig:log_ll}
\end{figure}

The parameters of the Vicsek model inferred in this fitting are shown in
\cref{fig:lukeman_params}. The tendency of the interaction radius $r$ towards
zero suggests that there is no evidence of a direct alignment behaviour between
individuals of this flock. The flock's highly polarized structure must then be
the result of behaviours unaccounted for by our model: such as attraction or
repulsion interactions. With no effective interaction term, all the observed
directional changes are accounted for by the inferred noise parameters. The
small degrees of freedom parameter inferred gives evidence of non-normally
distributed directional changes.

\begin{figure}[tb]
  \includegraphics{params_hist.pdf}
  \caption{Samples drawn from the posterior distribution realised in fitting a
    sequence of the Lukeman dataset with the \emph{least} amount of missingness
    to a variation of the Vicsek model. Prior beliefs are overlain in red.}
  \label{fig:lukeman_params}
\end{figure}

In \cref{fig:dir_hist} we see histogram plots of samples realised
from the posterior: here representing four directions missing at the end of the
sequence. We see that the uncertainty in the possible directions of motion
of the missing agent is large. The large posterior variance is a product of the
large amounts of missingness displayed by this sequence, and the large degrees
of freedom parameter inferred (\cref{fig:lukeman_params}).

\begin{figure}[tb]
  \includegraphics[width=0.9\textwidth]{dir_hist.pdf}
  \caption{Histogram plots of posterior draws of $4$ missing directions (out of
    a total of $680$ missing data points). The posterior draws visualised
    correspond to data missing at the end of the sequence. We see large amounts
    of uncertainty in these posterior densities, reflecting the large amount of
    missingess represented by this sequence.}
  \label{fig:dir_hist}
\end{figure}

\section*{Conclusions}

In this chapter we considered a flocking event in which at least one individual
was out of view during every frame of the recorded event. Of the $4577$ data
points in the analysed sequence, $680$ ($\approx16\%$) of the observations were
missing. To evaluate the likelihood of this data we had to be able to account
for the position and direction of every individual in the analysed frames. As
at least one individual is missing in every frame, discarding frames containing
missingness is not a viable solution. Instead we account for the missingness
using methodology outlined in \cref{cha:missing}.

The large amount of missing data in the analysed sequence resulted in a
high-dimensional posterior distribution, and hence a computationally intensive
inference problem. Long simulations were necessary to allow the simulated
Markov chains to achieve convergence. Although the inference scheme converged,
our results showed evidence that there is no explicit alignment interaction
between individuals, suggesting the presence of interactions not accounted for
by our model. As our model couldn't capture these interactions, it instead
attempted to account for directional changes through noise alone (in a similar
manner to the null model), which in turn resulted in large uncertainties about
missing positions and directions of motion.
