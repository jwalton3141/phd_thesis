\graphicspath{{fig/sheep/}}

\chapter{Case study I: flocking sheep}
\label{cha:sheep}

In \cref{cha:sim_studies} we showed that it is possible to infer the parameters
of a number of agent-based models from simulated data. Performing simulation
studies before attempting inference on real data is advisable as it allows an
opportunity to troubleshoot and assess the accuracy of our inference schemes.
The schemes constructed to perform inference on simulated data can now be
reused with little-to-no-alteration to perform inference on real data. With the
confidence instilled by the success of our simulation studies, if we find our
schemes unable to fit a model to \emph{real} data, we can be confident that
this is because of a discrepancy between the model and data, rather than an
error in our fitting process or our implementation of it.

This chapter shall focus on data of flocking sheep, unseen before in the
literature. Aerial footage of flocking sheep was recorded by a commercially
available drone. The position of each individual sheep was extracted from the
captured video footage using custom tracking software. In total, three flocking
sequences were extracted and analysed. The flocking events were captured and
their data extracted by Hayley Moore (thesis in preparation). Global models
considered in \cref{sec:global_models} will be fit to these sequences. The
resulting fits will be assessed for validity and ranked for performance by
Akaike Information Criteria.

\section{Flocking data}
\label{sec:data}

Aerial footage of flocking sheep was captured with a commercially available DJI
Phantom $3$ drone equipped with a built in high-definition camera, representing
a frame resolution of $4000\times3000$ pixels. This video footage was recorded
at a rate of $24$ frames per second. In total, three distinct flocking events
were captured and analysed. Each flocking event involved the same flock of $45$
sheep.

The recorded flocking events took place in large open fields: a natural
environment with which the sheep were already familiar. Flocks were recorded in
familiar terrain in an attempt to observe the most natural flocking events, and
to minimise the effect which an unfamiliar environment may have on the flock
dynamics. In some cases the flocking events were initiated by the movements of
a quad bike, but were then left to develop naturally and unprompted. Events
generally took place away from hard boundaries, such as fences and trees, but
the influence of these objects from a distance cannot be ruled-out.

Custom tracking software was constructed to determine the position of each
sheep in every video frame. To correct for the movement of the drone due to
external influences, such as wind, the resulting frames were transformed with
respect to some reference points. Reference points are stationary and
identifiable features present in all of the captured frames. Example reference
points include field boundaries, fence posts and trees.

Having corrected for extraneous drone-movement, the captured frames were then
thresholded to separate the pixels representing sheep from their surrounding
environment. Kalman filtering and the Hungarian algorithm were then used to
track the positions of individual sheep between frames. Having linked the
positions of sheep between frames, the trajectories of motion of every
individual could be reconstructed. \cref{fig:sheep_frame} shows the positions
of sheep in a single video frame, overlain are the positions of these sheep in
previous frames.

\begin{figure}[tbp]
  \includegraphics[width=0.8\textwidth]{traj0130_crop.png}
  \caption{A visualisation of data realised from a single flocking event. The
    faint vertical line seen passing through the image shows a boundary fence
    which was located here previously. The position of each sheep in every
    video frame was extracted using custom tracking software. Coloured lines
    represent the positions of sheep in previous frames. Each flocking event
    tracked the movements of $45$ sheep over $\approx200$ frames. A single
    flocking sequence then represents approximately $45\times200=9000$
    observations. Footage was recorded at a rate of 24 frames per second, and so
    each sequence represents $\approx10$\,seconds of raw footage.}
  \label{fig:sheep_frame}
\end{figure}

Having realised the trajectories of each individual it is then a simple process
to determine the velocity, speed and direction of motion of each individual.
Using \cref{eq:alignment} it is possible to compute the mean polarisation of
the flocks in each sequence. The attributes of each flocking event are
summarised in \cref{tab:data_summary}. We see that sequences are broadly
similar: representing highly-polarised flocks moving at similar speeds.

\begin{table}[tbp]
\begin{tabular}{@{}crrrr@{}}
\toprule
Sequence & Frames & Sheep & Mean speed (px\,s$^{-1}$) & Mean alignment \\
\midrule
1 &    192 &     45 &      4.20 &          0.99 \\
2 &    183 &     45 &      3.26 &          0.98 \\
3 &    194 &     45 &      4.25 &          0.97 \\
\bottomrule
\end{tabular}
\caption{Summaries of the three flocking sequences analysed in this chapter.
  Each sequence involved a flock of the same $45$ sheep, recorded over
  $\approx200$ frames. As the data was recorded at a rate of $24$ frames per
  second, each sequence represents around $10$ seconds of raw footage. The
  observed flocks were all highly polarized and moved at similar speeds.}
\label{tab:data_summary}
\end{table}

\cref{fig:seq_1_traj,fig:seq_2_traj,fig:seq_3_traj} show the trajectories of
motion of each individual in the three recorded flocking events. The positions
of sheep in the first frame of the captured footage are represented by red
markers. The sheep then travel along the blue lines, with their last recorded
position illustrated by the green markers.

\begin{figure}[tbp]
  \includegraphics{seq_1_traj.pdf}
  \caption{The trajectories of sheep reconstructed from video footage of
    sequence $1$. The realised flock shows $45$ individuals moving cohesively
    over $192$ frames. The observations in the first frame of this sequence were
    used to seed the forward simulations in \cref{cha:sim_studies}.}
  \label{fig:seq_1_traj}
\end{figure}
\begin{figure}[tbp]
  \includegraphics{seq_2_traj.pdf}
  \caption{A trajectory plot of the flock realised in sequence $2$. These
    trajectories correspond to those shown in the drone footage of
  \cref{fig:sheep_frame}.}
  \label{fig:seq_2_traj}
\end{figure}
\begin{figure}[tbp]
  \includegraphics{seq_3_traj.pdf}
  \caption{Illustrating the movement of the flock in sequence $3$. Although
    this flock appears as cohesive as the events in sequences $1$ and $2$,
    this flock is also seen to go through greater directional changes.}
  \label{fig:seq_3_traj}
\end{figure}

The data realised by the drone captured footage resembles the data used in the
simulation studies of \cref{cha:sim_studies}. In fact, the first observed frame
of sequence $1$ was used to seed these forward simulations.
    
\section{Model fitting}

We now proceed to fit the global agent-based models introduced in
\cref{cha:model_dev} to the three flocking events realised. The performance of
these fitted models is then assessed by the Akaike Information Criteria. Model
adequacy is inspected with standardised residual plots and posterior predictive
checking.

\subsection{Sequence 1}

Sequence $1$ represents the most cohesive flocking event of the captured
sequences. We shall use the inference schemes constructed in
\cref{cha:sim_studies} to fit our candidate models to this data. With the
models fitted, we can then assess the performance of the model fits, and rank
the candidate models by their predictive performance.

\subsubsection{Posterior beliefs}

Our posterior beliefs about the parameters of the candidate models fitted to
sequence $1$ are shown in \cref{fig:posteriors_seq1}. The same vague prior
beliefs that were specified in \cref{cha:sim_studies} were reused here. These
prior beliefs are plotted in red on top of our posteriors. For all the inferred
parameters our prior beliefs appear flat in comparison to our posterior
beliefs. This shows that we were able to learn a lot from the data, and that
our prior beliefs did not have a strong impact on our posterior realisations.

Stan's No-U-Turn-Sampler was used to draw samples from the posterior
distributions of the Null, power-law weighted, Gaussian weighted and
topological models. To perform the model fittings we initialised four
independent Markov chains at draws from our prior beliefs. Chains were
simulated for $10,000$ iterations, with the first $5000$ iterations discarded
to allow convergence. The samples drawn by each independent chain are
represented by the different coloured histograms of \cref{fig:posteriors_seq1}.
The overlain histograms show that the chains all converged to the same common
distribution. Computing values of $\widehat{R}$ also indicate that our chains
have converged.

The parameters of the Vicsek model were inferred by implementing a random walk
Metropolis--Hastings sampler. This algorithm was simulated for $10^6$
iterations, with the first half of the chains discarded to allow a burn-in
period. Our posterior distributions about the interaction radius $r$, shown in
\cref{subfig:posterior_seq1_vicsek}, show a very strong negative skew. As our
posterior density rapidly drops to zero as $r\rightarrow3.25$, we conclude  
that at $r\approx3.25$ an agent suddenly includes the influence of a neighbour
which the model finds incompatible. This behaviour is a product of the
discontinuity of Vicsek's weighting rule.

The interaction parameters inferred from sequence $1$ all represent weak
interactions. This finding is consistent with the observation that weak
alignment interactions are sufficient to maintain groups which are already
highly polarised \parencite{jhawar20}. The small values inferred for the
parameter $\nu$, representing the degrees of freedom of the Student's
$t$-distribution, suggests evidence of non-normality of the noise. This result
is significant as thus far the assumption of normally distributed noise has
been a mainstay of agent-based modelling.

Although we have shown that our candidate models \emph{can} be fit to sequence
$1$, this does not say anything about their adequacy or goodness-of-fit. To
assess goodness-of-fit we shall inspect residual plots and make posterior
predictive checks.

\begin{figure}[p]
  \captionsetup[subfigure]{aboveskip=0pt,
                           belowskip=6pt,
                           oneside,
                           margin={0.7cm,0cm}}
  \begin{subfigure}[b]{\textwidth}
    \centering
    \includegraphics{seq1/null_hist_sigma_Y.pdf}%
    \includegraphics{seq1/null_hist_nu.pdf}
    \caption{Null model}
    \label{subfig:posterior_seq1_null}
  \end{subfigure}
  \begin{subfigure}[b]{\textwidth}
    \includegraphics{seq1/r_hist_r.pdf}%
    \includegraphics{seq1/r_hist_sigma_Y.pdf}%
    \includegraphics{seq1/r_hist_nu.pdf}
    \caption{Vicsek model}
    \label{subfig:posterior_seq1_vicsek}
  \end{subfigure}
  \begin{subfigure}[b]{\textwidth}
    \includegraphics{seq1/power_hist_alpha.pdf}%
    \includegraphics{seq1/power_hist_sigma_Y.pdf}%
    \includegraphics{seq1/power_hist_nu.pdf}
    \caption{Power-law weighted model}
    \label{subfig:posterior_seq1_power}
  \end{subfigure}
  \begin{subfigure}[b]{\textwidth}
    \includegraphics{seq1/gauss_hist_sigma_X.pdf}%
    \includegraphics{seq1/gauss_hist_sigma_Y.pdf}%
    \includegraphics{seq1/gauss_hist_nu.pdf}
    \caption{Gaussian weighted model}
    \label{subfig:posterior_seq1_gauss}
  \end{subfigure}
  \begin{subfigure}[b]{\textwidth}
    \includegraphics{seq1/top_hist_k.pdf}%
    \includegraphics{seq1/top_hist_sigma_Y.pdf}%
    \includegraphics{seq1/top_hist_nu.pdf}%
    \caption{Topological model}
    \label{subfig:posterior_seq1_top}
  \end{subfigure}
  \vspace{-1.5em}
  \caption{Histogram plots show the posterior samples realised in fitting
    candidate models to the data of sequence $1$. Our priors beliefs are overlain
    in red, and in all cases appear flat in comparison to our posterior
    beliefs; this indicates that the observed data has been very informative.}
  \label{fig:posteriors_seq1}
\end{figure}

\subsubsection{Standardised residuals}

Assessing plots of residuals can be used as an informal method to assess
modelling assumptions. To compare model and residual, we first rearrange
\cref{eq:students_update} to see that:
\begin{equation}
  \label{eq:residuals}
  (\theta_{i,t+1} - \angmean{\theta}_{i,t}) / \sigma_Y \given \nu \sim t_{\nu},
\end{equation}
where $t_{\nu}$ is the standardised Student's $t$-distribution with $\nu$
degrees of freedom. To realise a model's residuals we compute
$\angmean{\theta}_{i,t}$ for $i=1,\ldots,N$ and $t=1,\ldots,T$ for every
posterior sample made by our inference scheme. The residuals can then be
computed as $\theta_{i,t+1} - \angmean{\theta}_{i,t}$ and standardised by
dividing by the posterior mean of the scale parameter $\sigma_Y$. The residuals
are compared with the model graphically by overlaying a standardised Student's
$t$-distribution with the degrees of freedom parameter taken from our posterior
mean.

\cref{fig:residuals_seq1} shows the standardised residuals of our fitted
models. We see that all the models look to give a good fit to the data. It may
come as a surprise that the Null model looks to perform so well. However, this
is again a product of the observation that our flock was already highly
polarised, and that only weak alignment interactions are necessary to maintain
a flock which is already cohesive.

\begin{figure}[tbb]
  \captionsetup[subfigure]{aboveskip=2pt,
                           belowskip=9pt,
                           oneside,
                           margin={0.7cm,0cm}}
  \begin{subfigure}[t]{0.33333\textwidth}
    \caption{Null model}
    \includegraphics{seq1/null_residuals.pdf}
  \end{subfigure}\hspace{2pt}
  \begin{subfigure}[t]{0.33333\textwidth}
    \caption{Vicsek model}
    \includegraphics{seq1/r_residuals.pdf}
  \end{subfigure}\vspace{1em}\\
  \begin{subfigure}[t]{0.33333\textwidth}
    \includegraphics{seq1/power_residuals.pdf}
    \caption{Power-law weighted model}
  \end{subfigure}%
  \begin{subfigure}[t]{0.33333\textwidth}
    \includegraphics{seq1/gauss_residuals.pdf}
    \caption{Gaussian weighted model}
  \end{subfigure}%
  \begin{subfigure}[t]{0.33333\textwidth}
    \includegraphics{seq1/top_residuals.pdf}
    \caption{Topological model}
  \end{subfigure}
  \caption{Standardised residuals of the models fitted to sequence $1$. The
  residuals are standardised by dividing by the posterior mean of the scale
  parameter $\sigma_Y$. A Student's $t$-distribution with $\nu$ degrees of
  freedom taken from our posterior mean is overlain in green. We see that
  models all look to provide a good fit to the data.}
  \label{fig:residuals_seq1}
\end{figure}

Although plots of standardised residuals provide a useful informal measure of
model fit, it can be useful to use more formal methods such as information
criteria to quantify fit.

\subsubsection{Model rankings}

Akaike Information Criteria (AIC) favours models which better explain the
given data, expressed through a larger likelihood, but penalises model
complexity---quantified as the number of model parameters. The Akaike weight of
a model can be computed from its AIC value. Typically a model's Akaike weight is
interpreted as the probability that the model will make the best predictions on
new data, conditional on the set of models considered.

In \cref{tab:aic_seq1} we show the value of AIC computed for each model fit.
Recall that \emph{smaller} values of AIC indicate a \emph{better} estimate of
predictive performance. \cref{tab:aic_seq1} shows the ranking of each model
fit. We see that the topological model was ranked as the best performing model,
followed by the power-law and Gaussian weighted models. The Null and Vicsek
models are ranked as the worst performing. If we were to only compare maximised
likelihoods we would rank the Null and Vicsek models the same. However, AIC
penalises for model complexity, and so having an extra model parameter without
being better at describing the data: the Vicsek model is ranked below the Null
model.

\begin{table}[tbp]
\begin{tabular}{@{}lrrrrr@{}}
\toprule
Model                       &      AIC & rank & pAIC & dAIC & weight \\
\midrule
Topological                 & -49\,031 &    1 &  3 &  0.0 &   0.98 \\
Power-law                   & -49\,023 &    2 &  3 &  7.4 &   0.02 \\
Gaussian                    & -48\,985 &    3 &  3 & 46.3 &   0.00 \\
Null                        & -48\,943 &    4 &  2 & 87.7 &   0.00 \\
Vicsek                      & -48\,941 &    5 &  3 & 89.7 &   0.00 \\
\bottomrule
\end{tabular}
\caption{Tabulated values of AIC and the corresponding Akaike weights
  (displayed to $2$ decimal places) of each model fitted to sequence $1$. Models
  are ranked from $1$--$5$, where $1$ is the best fit and $5$ is the worst.}
\label{tab:aic_seq1}
\end{table}

\subsubsection{Posterior predictive checks}

Forward simulating candidate models with parameters drawn from our posterior
beliefs provides an informal method to assess how well a model can reproduce
realistic behaviours. Here we forward simulate each candidate model one
thousand times. To initialise each simulation we randomly draw model parameters
from our posterior density. We simulate the movement of $45$ sheep, initially
positioned and directed as in the first frame of the fitted sequence, for $192$
time steps. We compute the alignment of the flocks at each time step.

\cref{fig:checks_seq1} shows the alignment of the simulated flocks. The top
panel shows the alignment of two sets of randomly selected simulations. We see
that all the models do a reasonable job of capturing the observed flock
alignment for the first $\approx25$ time steps. After this, the observed flock
experiences a fluctuation in alignment over a time scale of $\approx100$
frames. The candidate models do not appear capable of producing flocks with
such long-time fluctuations in alignment. However, this should not come as a
particular surprise as these models only predict direction of motion one step
ahead. The disparity between observed alignment and simulated alignment may
suggest the presence of interactions or external influences not accounted for
by our models, or of a longer range dependence on previous directions.

\begin{figure}[tbp]
  \includegraphics{alignment/alignment_single_1.pdf}\vspace{1em}\\
  \includegraphics{alignment/alignment_ensemble_1.pdf}
  \caption{The alignment of flocks forward simulated with parameters drawn from
    our posterior densities about model fits to sequence $1$. Simulations were
    initialised as in the first frame of the observed data, and repeated one
    thousand times. The top panels show two randomly selected sets of forward
    simulations. The bottom panel shows the median alignment across all
    simulations. Bands represent the upper and lower quartiles over the one
    thousand repetitions.}
  \label{fig:checks_seq1}
\end{figure}

The bottom panel of \cref{fig:checks_seq1} shows the median alignment of
simulated flocks as a function of time. The coloured bands around the medians
show the upper and lower quartiles of the realised alignments. We see that the
topological and Gaussian models, favoured by AIC, produce the most consistently
cohesive flocks.

\subsection{Sequence 2}

This sequence details the movements of our flock of $45$ sheep over $183$
frames. The recorded flock is again very coherent and highly polarised. 

\subsubsection{Posterior beliefs}

Parameter inference is again performed by a combination of the Metropolis--Hastings
and NUTS algorithms. The posterior realisations made by these samplers are shown
in \cref{fig:posteriors_seq2}. Having overlain our prior densities in red,
we can see that our beliefs have changed considerably in light of the data.

The inferred parameters are similar to those realised from sequence $1$;
representing weak interactions between individuals and giving evidence of
agents experiencing non-normally distributed noise. Similar noise-scale and
degrees of freedom parameters are inferred for every model fitted to sequence
$2$.

Our posterior densities about the interaction radius of the Vicsek model again
shows a strongly skewed distribution. It is this sudden drop in posterior
density which makes the NUTS algorithm ineffective for this problem.

\begin{figure}[p]
  \captionsetup[subfigure]{aboveskip=1pt,
                           belowskip=8pt,
                           oneside,
                           margin={0.7cm,0cm}}
  \begin{subfigure}[b]{\textwidth}
    \centering
    \includegraphics{seq2/null_hist_sigma_Y.pdf}\hspace{0.01\textwidth}
    \includegraphics{seq2/null_hist_nu.pdf}
    \caption{Null model}
  \end{subfigure}
  \begin{subfigure}[b]{\textwidth}
    \includegraphics{seq2/r_hist_r.pdf}%
    \includegraphics{seq2/r_hist_sigma_Y.pdf}%
    \includegraphics{seq2/r_hist_nu.pdf}
    \caption{Vicsek model}
  \end{subfigure}
  \begin{subfigure}[b]{\textwidth}
    \includegraphics{seq2/power_hist_alpha.pdf}%
    \includegraphics{seq2/power_hist_sigma_Y.pdf}%
    \includegraphics{seq2/power_hist_nu.pdf}
    \caption{Power-law weighted model}
  \end{subfigure}
  \begin{subfigure}[b]{\textwidth}
    \includegraphics{seq2/gauss_hist_sigma_X.pdf}%
    \includegraphics{seq2/gauss_hist_sigma_Y.pdf}%
    \includegraphics{seq2/gauss_hist_nu.pdf}
    \caption{Gaussian weighted model}
  \end{subfigure}
  \begin{subfigure}[b]{\textwidth}
    \includegraphics{seq2/top_hist_k.pdf}%
    \includegraphics{seq2/top_hist_sigma_Y.pdf}%
    \includegraphics{seq2/top_hist_nu.pdf}%
    \caption{Topological model}
  \end{subfigure}
  \vspace{-2em}
  \caption{Posterior realisations of the model parameters inferred in fitting
  candidate models to sequence $2$. Posterior samples of the parameters of the
  Vicsek model were drawn by the implementation of a Metropolis--Hastings
  sampler. The remaining models were fitted with the implementation of Stan's
  NUTS algorithm.}
  \label{fig:posteriors_seq2}
\end{figure}

\subsubsection{Standardised residuals}

We use plots of standardised residuals to compare observation and model
prediction. From \cref{eq:residuals} we expect the standardised difference
between prediction and observation to follow a Student's $t$-distribution with
$\nu$ degrees of freedom. Our observations are the directions of motion
$\theta_{i,t+1}$, and our predictions are given by the interaction
$\angmean{\theta}_{i,t}$ computed at each posterior sample.

The standardised residuals computed from each model fit are shown in
\cref{fig:residuals_seq2}. The standardised residuals look to follow a
generalised Student's $t$-distribution with scale $1$, and location slightly
below zero. The non-zero mean of the residuals may suggest the presence of
additional interactions such as attraction or repulsion, which are not
implemented in our models. The non-zero mean could also be explained by the
influence of external factors, such as field boundaries, which are not
accounted for by our models.

The residual plots for each model fit look very similar. This is again a
product of the observation that our flocks are already highly aligned,
experience only small fluctuations in directional change, and only require weak
alignment interactions to maintain cohesion.

\begin{figure}[tbp]
  \captionsetup[subfigure]{aboveskip=2pt,
                           belowskip=9pt,
                           oneside,
                           margin={0.7cm,0cm}}
  \begin{subfigure}[t]{0.33333\textwidth}
    \caption{Null model}
    \includegraphics{seq2/null_residuals.pdf}
  \end{subfigure}\hspace{2pt}
  \begin{subfigure}[t]{0.33333\textwidth}
    \caption{Vicsek model}
    \includegraphics{seq2/r_residuals.pdf}
  \end{subfigure}\vspace{1em}\\
  \begin{subfigure}[t]{0.33333\textwidth}
    \includegraphics{seq2/power_residuals.pdf}
    \caption{Power-law weighted model}
  \end{subfigure}%
  \begin{subfigure}[t]{0.33333\textwidth}
    \includegraphics{seq2/gauss_residuals.pdf}
    \caption{Gaussian weighted model}
  \end{subfigure}%
  \begin{subfigure}[t]{0.33333\textwidth}
    \includegraphics{seq2/top_residuals.pdf}
    \caption{Topological model}
  \end{subfigure}
  \caption{Standardised residuals of models fitted to sequence $2$. The density
    overlain in green represents a Student's $t$-distribution with $\nu$ degrees
    of freedom, where $\nu$ is taken as the posterior mean from the corresponding
    model fit. The distributions of residuals show a mean slightly below zero,
    suggesting the presence of behaviours not accounted for in our models.}
  \label{fig:residuals_seq2}
\end{figure}

\subsubsection{Model rankings}

The standardised residuals of each model fit shown in \cref{fig:residuals_seq2}
look very similar. Recall that model predictions are made at every iteration of
our inference schemes. For samples drawn using Stan we have $4$ independent
Markov chains, simulated for $5000$ post-warm-up iterations, giving a total of
$20,000$ iterations. Sequence $2$ contains $45\times183$ data points. Given the
initial directions of individuals, we are then tasked with predicting
$45\times182=8190$ data points. Making our predictions at every iteration, we
make a total of $20,000\times8190$ predictions per model. These predictions are
then used to form our standardised residuals. As such, it can be very difficult
to visually ascertain small improvements in model performance from plots of
residuals alone. Because of this, more quantitative measures such as AIC can be
useful in distinguishing between models.

Computed AIC values and their corresponding model ranks are tabulated in
\cref{tab:aic_seq2}. The computed Akaike weights are not as conclusive as they
were with sequence $1$; here the power-law and topological models are awarded
similar Akaike weights. The similarity in weights computed for these models
suggests that in application there may be very little practical distinction
between them, or that their differences are so minute it can be difficult to
tell them apart.

Below the ranking of the power-law weighted and topological models, the ranking
follows the same order as was seen in sequence $1$. Again we see that the
Vicsek model was not able to explain any more of the data than the Null model.
But as AIC favours simplicity, given that both models have the same value of
the maximised likelihood function, we see the Vicsek model ranked below the
Null model.

\begin{table}[tbp]
\begin{tabular}{@{}lrrrrr@{}}
\toprule
Model                       &        AIC & rank & pAIC & dAIC & weight \\
\midrule
Power-law                   & -43\,495 &    1 &  3 &  0.0 &   0.49 \\
Topological                 & -43\,494 &    2 &  3 &  0.2 &   0.44 \\
Gaussian                    & -43\,490 &    3 &  3 &  4.5 &   0.05 \\
Null                        & -43\,488 &    4 &  2 &  6.9 &   0.02 \\
Vicsek                      & -43\,486 &    5 &  3 &  8.9 &   0.01 \\
\bottomrule
\end{tabular}
\caption{Ranking the models fitted to sequence $2$ by their predictive
  performance, estimated by AIC. The power-law model is ranked as the best
  performing model, followed closely by the topological model.}
\label{tab:aic_seq2}
\end{table}

\subsubsection{Posterior predictive checks}

To assess the ability of our fitted models to produce realistic looking
behaviours, we forward simulate the candidate models using parameters drawn
from our posterior densities. Simulations are initialised as in the first frame
of sequence $2$, and repeated one thousand times. For each repetition new
parameter values are drawn from the posterior.

\cref{fig:checks_seq2} shows alignment as a function of time for the resulting
flocks. The top panel of \cref{fig:checks_seq2} shows two sets of randomly
selected flocking instances. Here the models look to provide a good prediction 
of flock alignment for the first $\approx75$ time steps. The bottom panel of
\cref{fig:checks_seq2} shows the median flock alignment as a function of time.
The bands around the coloured lines represent the $25$-th and $75$-th
percentiles of the observed alignments. The most cohesive flocks were generated
by the topological and power-law models---again corresponding to the models
favoured by AIC.

\begin{figure}[tbp]
  \includegraphics{alignment/alignment_single_2.pdf}\vspace{1em}\\
  \includegraphics{alignment/alignment_ensemble_2.pdf}
  \caption{Flock alignment as a function of time for forward simulated
    candidate models. Models are initialised from the first frame of sequence
    $2$, with model parameters drawn from our posteriors. Each candidate
    model is forward simulated one thousand times, with new model parameters
    drawn for each simulation. The top panel shows two randomly selected sets
    of forward simulations. The lower panel shows the median alignment of
    simulations as a function of time, with the bands around the median
    representing the $25$-th and $75$-th percentiles}
  \label{fig:checks_seq2}
\end{figure}

\subsection{Sequence 3}

The flocking event captured in sequence $3$ is much more dynamic than the
events captured in sequences $1$ and $2$: going through a much larger
directional change than the other sequences. In addition to this, the flock of
sequence $3$ is not as densely packed as the flocks of sequences $1$ and $2$,
as seen in \cref{fig:seq_3_traj}.

\subsubsection{Posterior beliefs}

Parameter values are inferred from sequence $3$ by sampling. The Null,
power-law weighted, Gaussian weighted and topological models are fit using
Stan, whereas the Vicsek model is fit with a Metropolis--Hastings algorithm.
The samples drawn from the posterior are shown in \cref{fig:posteriors_seq3}.
Computing values of $\widehat{R}$ shows that all our samplers converged.
Corresponding values of the effective sample size show a large number of
samples were realised from the posterior.

The inferred parameters broadly align with those inferred from sequences $1$
and $2$: with the inferred interaction parameters representing weak
interactions. However, the inferred degrees of freedom parameters are smaller
than those inferred for sequences $1$ and $2$, revealing a noise distribution
with more weight in its tails.

\begin{figure}[p]
  \captionsetup[subfigure]{aboveskip=0pt,
                           belowskip=9pt,
                           oneside,
                           margin={0.7cm,0cm}}
  \begin{subfigure}[b]{\textwidth}
    \centering
    \includegraphics{seq3/null_hist_sigma_Y.pdf}\hspace{0.01\textwidth}
    \includegraphics{seq3/null_hist_nu.pdf}
    \caption{Null model}
  \end{subfigure}
  \begin{subfigure}[b]{\textwidth}
    \includegraphics{seq3/r_hist_r.pdf}%
    \includegraphics{seq3/r_hist_sigma_Y.pdf}%
    \includegraphics{seq3/r_hist_nu.pdf}
    \caption{Vicsek model}
  \end{subfigure}
  \begin{subfigure}[b]{\textwidth}
    \includegraphics{seq3/power_hist_alpha.pdf}%
    \includegraphics{seq3/power_hist_sigma_Y.pdf}%
    \includegraphics{seq3/power_hist_nu.pdf}
    \caption{Power-law weighted model}
  \end{subfigure}
  \begin{subfigure}[b]{\textwidth}
    \includegraphics{seq3/gauss_hist_sigma_X.pdf}%
    \includegraphics{seq3/gauss_hist_sigma_Y.pdf}%
    \includegraphics{seq3/gauss_hist_nu.pdf}
    \caption{Gaussian weighted model}
  \end{subfigure}
  \begin{subfigure}[b]{\textwidth}
    \includegraphics{seq3/top_hist_k.pdf}%
    \includegraphics{seq3/top_hist_sigma_Y.pdf}%
    \includegraphics{seq3/top_hist_nu.pdf}%
    \caption{Topological model}
  \end{subfigure}
  \vspace{-2em}
  \caption{Posterior beliefs about the model parameters fitted to sequence $3$.
    The Null, power-law weighted, Gaussian weighted and topological models were
    fit with Stan's NUTS algorithm. The Vicsek model was fit with a
    Metropolis--Hastings algorithm. Prior beliefs are overlain in red, and
    appear flat in comparison to our posteriors.}
  \label{fig:posteriors_seq3}
\end{figure}

\subsubsection{Standardised residuals}

Residuals are computed as the difference between observation and model
prediction. Residuals are standardised by dividing by the posterior mean of the
inferred noise-scale parameter $\sigma_Y$. Histogram plots of the standardised
residuals are shown in \cref{fig:residuals_seq3}; a Student's $t$-distribution
with $\nu$ degrees of freedom, taken from our posterior mean, is overlain in
green. From \cref{fig:residuals_seq3} we see a poor fit of model to data:
although the residuals do look to follow a Student's $t$-distribution they do
so with a location less than zero. This suggests that we were unable to capture
the directional changes that the flock went through, and suggests the presence
of behaviours not included in our model. Such behaviours could be agent-agent
interactions such as attraction or repulsion, or could be interactions between
agents and their external environment.

\begin{figure}[tbp]
  \captionsetup[subfigure]{aboveskip=2pt,
                           belowskip=9pt,
                           oneside,
                           margin={0.7cm,0cm}}
  \begin{subfigure}[t]{0.33333\textwidth}
    \caption{Null model}
    \includegraphics{seq3/null_residuals.pdf}
  \end{subfigure}\hspace{2pt}
  \begin{subfigure}[t]{0.33333\textwidth}
    \caption{Vicsek model}
    \includegraphics{seq3/r_residuals.pdf}
  \end{subfigure}\vspace{1em}\\
  \begin{subfigure}[t]{0.33333\textwidth}
    \includegraphics{seq3/power_residuals.pdf}
    \caption{Power-law weighted model}
  \end{subfigure}%
  \begin{subfigure}[t]{0.33333\textwidth}
    \includegraphics{seq3/gauss_residuals.pdf}
    \caption{Gaussian weighted model}
  \end{subfigure}%
  \begin{subfigure}[t]{0.33333\textwidth}
    \includegraphics{seq3/top_residuals.pdf}
    \caption{Topological model}
  \end{subfigure}
  \caption{Standardised residuals from sequence $3$ show a poor fit of model to
    data. Although the residuals do look to follow a Student's $t$-distribution,
    they do so with a location less than zero. This suggests the presence of
    behaviours not captured by our models. Missing behaviours could be
    interactions between sheep and their external environment, or missing
    sheep-sheep interactions such as repulsion or attraction behaviours.}
  \label{fig:residuals_seq3}
\end{figure}

\subsubsection{Model rankings}

Although plots of the standardised residuals show that none of the considered
models provide a particularly good fit to data, it can still be informative to
consider if any of the models outperformed the others. \cref{tab:aic_seq3}
shows the AIC values and corresponding Akaike weights computed for each model
fit. From this we see that the power-law weighted provided the best fit to the
data. Consistent with sequences $1$ and $2$ we see the Null and Vicsek models
as the worst performing candidate models.

\begin{table}[tbp]
\begin{tabular}{@{}lrrrrr@{}}
\toprule
Model                       &      AIC & rank & pAIC &  dAIC & weight \\
\midrule
Power-law                   & -52\,242 &    1 &  3 &   0.0 &   1.00 \\
Gaussian                    & -52\,222 &    2 &  3 &  20.3 &   0.00 \\
Topological                 & -52\,169 &    3 &  3 &  73.6 &   0.00 \\
Null                        & -52\,037 &    4 &  2 & 205.8 &   0.00 \\
Vicsek                      & -52\,035 &    5 &  3 & 207.8 &   0.00 \\
\bottomrule
\end{tabular}
\caption{Estimating predictive performance of models fitted to sequence $3$
  using Akaike Information Criteria. Models are ranked from best performing ($1$)
  to worst performing ($5$). The corresponding Akaike weight for each model is
  computed.}
\label{tab:aic_seq3}
\end{table}

\subsubsection{Posterior predictive checks}

Forward simulating fitted models provides an opportunity to assess a model's
ability to reproduce realistic behaviours. Here we assess the ability of our
models to replicate the change in alignment over time of our data. Although our
residual plots show that we were unable to explain the directional changes of
sequence $3$, \cref{fig:checks_seq3} shows that out models do a reasonable job
of mimicking the alignment of sequence $3$ over time. The bottom panel of
\cref{fig:checks_seq3} shows that the power-law weighted model produces the most
consistently cohesive flocks, which was also the model favoured by AIC.

\begin{figure}[tbp]
  \includegraphics{alignment/alignment_single_3.pdf}\vspace{1em}\\
  \includegraphics{alignment/alignment_ensemble_3.pdf}
  \caption{Alignment as a function of time for forward simulations of sequence
    $3$. Model parameters were chosen as random draws from our posterior
    beliefs. Each candidate model was forward simulated one thousand times,
    with new parameter values drawn for each simulation. The top panel shows two
    sets of randomly selected forward simulations. The lower panel shows the
    median alignment of flocks as a function of time, the bands around the
    median represent the upper and lower quartiles of the alignment.}
  \label{fig:checks_seq3}
\end{figure}

\subsection*{Conclusions}

We have evidenced that we are able to fit a number of competing agent-based
models to real flocking events, working in a Bayesian framework. We
demonstrated a number of different ways to assess model fit, and estimate
predictive performance. Although here we only focussed on a small subset of
possible candidate models, our approach could be extended to fit and compare
any number of models.

Our inferred posteriors consistently showed that only weak interaction
parameters are necessary to maintain flocking events which are already
highly-polarised. The small values of the degrees of freedom parameters
inferred gave evidence of non-normally distributed noise.

Assessing standardised residual plots showed that the models generally
provided a good fit to the data. However, for one of our sequences we
observed that our models were unable to capture all the behaviours of the
flock. From this we concluded that additional interaction rules, such as
attraction or repulsion behaviours, or interactions with the environment,
needed to be included.

In ranking our model fits we observe that the power-law weighted and
topological models perform the best. That no single model comes out on top
suggests that there may be aspects to the data which neither model is able to
capture, or that the practical distinction between these models is so small
that it is difficult to tell them apart. Though our analysis does not favour a
single model as providing the best fit, models implementing continuous
interaction rules consistently outrank those with discontinuous rules. We saw
that the Vicsek model is no better at describing the data than the Null model,
even with the additional complexity of an extra model parameter. Penalised for
complexity, AIC consistently ranked the Vicsek model below the null model;
making the Vicsek model the worst performing candidate model.

