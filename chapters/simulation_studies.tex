\graphicspath{{fig/sim_studies/}}

\chapter{Simulation studies}
\label{cha:sim_studies}

The feasibility of performing parameter inference on real data can be assessed by
simulation study. In simulation studies, candidate models are forward simulated for known
parameter values. This simulated data is then used to verify that the known parameter
values can be recovered by inference.

If parameter inference is not possible on simulated data, we know that the same inference
will not be possible on real data. Simulation studies also provide good opportunity to
assess the implementation of MCMC algorithms intended to target the posterior
distribution.

\section{Global models}

Global models refer to those in which all agents take the same parameter values. These are
in contrast to hierarchical models, in which all agents take different parameter values.
We shall first seek to perform simulation studies on these global models, as they
represent a simpler modelling paradigm.  

Recall from Bayes' Theorem that to target the posterior distribution we need to combine
the likelihood of the data with prior beliefs. Let us begin by considering the likelihood
of observing a single agent updating its direction, in the presence of neighbours, from
$\theta_{i, t}$ to $\theta_{i, t+1}$. Considering that agent $i$ experiences noise
distributed according to a generalised Students $t$-distribution with $\nu$ degrees of
freedom (\cref{eq:students_update}), we realise this likelihood as:
\begin{equation*}
    L(\sigma_Y, \nu, \angmean{\theta}_{i,t} \given \theta_{i,t+1}, \theta_{i,t}) = 
    \frac{\Gamma(\frac{\nu +1 }{2})}{\Gamma(\frac{\nu}{2})\sqrt{\pi\nu}\sigma_Y}
        \Bigg(1 + \frac{1}{\nu}
                  \bigg(\frac{\theta_{i,t+1}-\angmean{\theta}_{i,t}}{\sigma_Y}\bigg)^2
        \Bigg)^{-\frac{\nu+1}{2}},
\end{equation*}
where $\Gamma$ is the gamma function. Building on this, let us consider the likelihood of
observing agent $i$'s directions at time $t$, $t+1$ and $t+2$. As successive noise terms
are independent, we may express this likelihood as a product:
\begin{equation*}
    L(\sigma_Y,\nu,\angmean{\theta}_{i,t},\angmean{\theta}_{i,t+1}\given\theta_{i,t+1},
    \theta_{i,t+2}) =
    L(\sigma_Y,\nu,\angmean{\theta}_{i,t+1}\given\theta_{i,t+2})\times
    L(\sigma_Y,\nu,\angmean{\theta}_{i,t}\given\theta_{i,t+1}).
\end{equation*}
In general, we wish to express the likelihood of observing agent $i$'s directional updates
from time $t=1$ to time $t=T$. Again, as realisations from the noise distribution are
independent, we may express this likelihood as a product:
\begin{align*}
    L(\sigma_Y, \nu, \angmean{\theta}_{i,1:T-1} \given \theta_{i,2:T})
    =& \prod_{t=1}^{T-1} L(\sigma_Y,\nu,\angmean{\theta}_{i,t} \given \theta_{i,t+1}) \\
    =& \prod_{t=1}^{T-1} 
        \frac{\Gamma(\frac{\nu +1 }{2})}{\Gamma(\frac{\nu}{2})\sqrt{\pi\nu}\sigma_Y}
        \Bigg(1 + \frac{1}{\nu}
                  \bigg(\frac{\theta_{i,t+1}-\angmean{\theta}_{i,t}}{\sigma_Y}\bigg)^2
        \Bigg)^{-\frac{\nu+1}{2}},
\end{align*}
where $\angmean{\theta}_{i,1:T-1}$ is shorthand for
$\angmean{\theta}_{i,1},\angmean{\theta}_{i,2},\ldots,\angmean{\theta}_{i,T-1}$, and
similarly for $\theta_{i,2:T}$. Finally, although we have expressed the likelihood of
observing a single agent's directional changes over $T$ observations, we wish to express
the likelihood of observing an entire flock's movements over $T$ time steps. Consider then
observing a flock of $N$ individuals observed at $T$ times. The likelihood of observing
this data is then given by
\begin{align*}
    L(\sigma_Y, \nu, \angmean{\theta}_{1:N,1:T-1} \given\theta_{1:N,2:T}) 
    &= \prod_{i=1}^N \prod_{t=1}^{T-1} L(\sigma_Y,\nu,\angmean{\theta}_{i,t}\given\theta_{i,t+1})  \\
    &= \prod_{i=1}^N \prod_{t=1}^{T-1} 
        \frac{\Gamma(\frac{\nu +1 }{2})}{\Gamma(\frac{\nu}{2})\sqrt{\pi\nu}\sigma_Y}
        \Bigg(1 + \frac{1}{\nu}
                  \bigg(\frac{\theta_{i,t+1}-\angmean{\theta}_{i,t}}{\sigma_Y}\bigg)^2
        \Bigg)^{-\frac{\nu+1}{2}}.
\end{align*}

\subsection{Vicsek interaction}

The NUTS algorithm implemented by Stan requires a continuously differentiable posterior
distribution. Unfortunately, the discontinuity in the weighting function of the
Vicsek model results in a discontinuous posterior. As such, Stan cannot be used to infer
the parameters of this model. Instead, we shall target the posterior distribution with a
random walk Metropolis--Hastings sampler.

\section{Hierarchical models}
